\documentclass[12pt,a4paper,colorlinks=true]{article}

\usepackage[left=1in, right=1in, top=1in, bottom=1in]{geometry}
\usepackage{bookmark}
\usepackage{graphicx}
\usepackage[export]{adjustbox}
\usepackage{subfig}
\usepackage{amsmath}
\DeclareMathOperator{\arccot}{arccot}

\usepackage{hyperref}
\hypersetup{ 
	colorlinks=true, 
	linkcolor=blue, 
	filecolor=blue, 
	citecolor = red,       
	urlcolor=cyan, 
} 

\usepackage{listings}
\usepackage{color} %red, green, blue, yellow, cyan, magenta, black, white
% \usepackage{xepersian}
% \settextfont{B Nazanin}


\definecolor{dkgreen}{rgb}{0,0.6,0}
\definecolor{gray}{rgb}{0.5,0.5,0.5}
\definecolor{mauve}{rgb}{0.58,0,0.82}

\lstset{frame=tb,
	language=MATLAB,
	aboveskip=3mm,
	belowskip=3mm,
	showstringspaces=false,
	columns=flexible,
	basicstyle={\small\ttfamily},
	numbers=none,
	numberstyle=\tiny\color{gray},
	keywordstyle=\color{blue},
	commentstyle=\color{dkgreen},
	stringstyle=\color{mauve},
	breaklines=true,
	breakatwhitespace=true,
	tabsize=3
}


\begin{document}
	
	
	\begin{center}
		\small In The Name Of God
	\end{center}
	\vspace{0.3in}
	\begin{figure}[h]
		\centering
		\includegraphics[width=2.8in]{SharifLogo.png}
		\label{SharifLogo}
	\end{figure}
	\begin{center}
		\textbf{\Huge Sharif University of Technology}\\
		\vspace{2mm}
		\LARGE Department of Electrical Engineering\\
		\vspace{10mm}
		\textbf{\LARGE Neuroscience Final Project}\\
		\vspace{4mm}
		\large by \\
		%\noindent\rule{10cm}{0.4pt}\\
		\large \textbf{Alireza Garegoori Motlagh}\\
		\large 9810----\\
		\large \&\\
		\large \textbf{Ali Nourian}\\
		\large 98102527\\
		\vspace{10mm}
		\Large Professor :\\
		\Large Dr. Karbalei Aghajan\\
		\vspace{20mm}
		\large semester 1399-2
		
	\end{center}

	\newpage
	
	{
		\hypersetup{linkcolor=black}
		\tableofcontents
	}
	
	\newpage
		
	\section{Introduction with Main Article} \label{S01}
	
	\subsection{The Main Goal of The Paper} \label{S01_1}
	
	An important goal in studying the receptive-field properties of
	visual neurons is to understand how they respond to complex
	spatiotemporal inputs, including those encountered in natural
	scenes.
	
	Several methods have been used to define
	basis sets for the efficient representation of visual stimuli, including
	principal component analysis (PCA), independent component
	analysis, and/or analysis based on sparse coding.
	
	For neurons with a linear stimulus–response relationship, relevant
	visual features can be identified by estimating their linear
	receptive fields using a spike-triggered average of the stimulus
	ensemble (also called “reverse correlation”) . This method has been widely used to measure the spatiotemporal receptive fields of neurons in the early visual
	pathway; the resulting receptive fields can largely account for the neuronal responses to complex spatiotemporal stimuli. However, in the visual cortex most of the neurons are complex cells with nonlinear stimulus–response relationships that
	cannot be characterized with the spike-triggered average.
	
	In the present study, we have used spike-triggered correlation analysis of
	the stimulus ensemble to construct the basis set for each complex cell.
	
	\subsection{Complex Neurons} \label{S01_2}
	
	Unit isolation was based on the
	cluster analysis of waveforms and the presence of a refractory period
	determined from the autocorrelograms. Cells were classified as simple if
	their receptive fields had clear on and off subregions and if the ratio of the first harmonic to the DC component of the
	response to an optimally oriented drifting grating was >1. All other cells were classified as complex. Among the 61 complex
	cells recorded, one was excluded from analysis because of its low firing
	rate in response to random-bar stimuli (<1 spike per second).
	
	\subsection{Spike-Triggered Correlations Analysis} \label{S01_3}
	
	In general, if certain features in the visual stimuli affect the firing probability of the cell, the spike-triggered stimulus ensemble should exhibit a different probability distribution from the entire stimulus ensemble.
	Although a change in the probability distribution can be reflected in a change in the first-order (mean), second-order (variance), or higher-order moments, the correlation analysis aims to identify features with changed variance. Because PCA results in a set of components with their variance ranking from the highest to the lowest, it is ideally suited for the identification of features with outstanding variance. Practically, identification of relevant features was achieved by finding eigenvalues of the spike-triggered correlation matrix that were significantly different from the eigenvalues of the control correlation matrix (computed by randomly sampling the entire stimulus ensemble).
	
	\newpage
	
	The spike-triggered correlation matrix was computed as follows:
	$$C_{mn} = \frac{1}{N} \sum_{i=1}^{N} S_m(i) S_n(i)$$
	where Sm(i) and Sn(i) are the mth and nth parameters of the stimulus pattern preceding the ith spike, respectively, and N is the total number of spikes in the response.
	Eigenvalues and eigenvectors of this spike-triggered correlation matrix were then computed. To compute each control correlation matrix, we generated a random spike train with the same number of spikes as in the recorded response but with random spike timing; the correlation matrix was computed based on this simulated random spike train.
	
	\subsection{Control Correlation Matrix} \label{S01_4}
	
	As we mentioned in the previous question, spike-triggered ensemble exhibits significantly different eigenvalues from the control correlation matrix. The article suggests that by using this idea and finding the projection of each trigger on the eigenvectors, as well as the projection of each spike-triggered ensemble on these eigenvectors, we can come up with a threshold to determine the type of a trigger.
	
	\subsection{Segregation Between Two Types of Visual Features} \label{S01_5}
	
	For most (47 of 60) of the complex cells studied, we found two significant eigenvectors that exhibit a much larger eigenvalue than others.

	These two eigenvectors exhibited separate on and off spatial subregions, resembling the receptive fields of simple cells. In a few cases (3 of 60), we found only one significant eigenvector for each complex cell; these vectors also exhibited spatiotemporal profiles resembling simple cell receptive fields. In the remaining cases, more than two eigenvalues reached significance. However, these additional eigenvectors (corresponding to third, fourth, ..., largest eigenvalues) tended to exhibit much less spatiotemporal structure than the first two eigenvectors, and their eigenvalues were much smaller, suggesting less functional importance
	

\begin{figure}[!hbt]
	\centering
	\begin{minipage}{.3\linewidth}
		\includegraphics[width = 5cm]{Q151.png}
		\caption{}
		\label{Q151.png}
	\end{minipage}%
	\begin{minipage}{.7\linewidth}
		\includegraphics[width = 12cm]{Q152.png}
		\caption{}
		\label{Q152.png}
	\end{minipage}
\end{figure}
	
	\newpage
		
	\section{Introduction with Dateset} \label{S02}
	
	
	\newpage
	
	\section{Spike-Trigger Average Method} \label{S03}
	
	\newpage
	
	\section{Spike-Trigger Correlation Method} \label{S04}
	
	\newpage
	
	\section{A Dirty Question:)} \label{S05}
	
	\newpage
	
	\section{appandix} \label{appandix}

	

	\[
	P(z) = 
	\begin{cases}
		P_a & z = a\\
		P_b & z = b\\
		0 & o.w.\\
	\end{cases}
	\]
	
	
		\[
	P(z) = 
	\begin{cases}
		\frac{a^bz^{b-1}}{(b-1)!}e^{-az} & z >= a\\
		0 & z < a
	\end{cases}
	\]
	

	\begin{equation} \label{gaussianEqu}
		g(x,y) = \frac{1}{2\pi\sigma^2}e^{-\frac{x^2+y^2}{2\sigma^2}}
	\end{equation}
	
	\[
	\begin{bmatrix}
	0.0005	&	0.0050	&	0.0109	&	0.0050	&	0.0005\\
	0.0050	&	0.0521	&	0.1139	&	0.0521	&	0.0050\\
	0.0109	&	0.1139	&	0.2487	&	0.1139	&	0.0109\\
	0.0050	&	0.0521	&	0.1139	&	0.0521	&	0.0050\\
	0.0005	&	0.0050	&	0.0109	&	0.0050	&	0.0005
	\end{bmatrix}
	\]

	
		\begin{table}[ht]
			\centering

			\begin{tabular}{|c|c|c|c|}
				\hline
				& Noisy Image & Filtered by Gaussian-filter & Filtered by Meadian-filter \\  [0.5ex]
				\hline 
				Gaussian    &   15.18     &   23.27    &	23.09  \\ 
				\hline 
				Poisson  &     22.22   &   26.96    &   26.78    \\ 
				\hline 
				Salt \& Pepper  &    12.68    &    21.78    &    29.44    \\ 
				\hline 
				Speckle  &    13.75    &   22.00    &    20.95    \\ 
				\hline 
			\end{tabular} 

			\caption{caption 0}
			\label{table1}
		\end{table}
	
	
	$$ MeanKernel = \frac{1}{N^2}I_{N\times N} $$
	
	\begin{figure}[!hbt]
		\centering
		\includegraphics[width = 7cm]{SharifLogo.png} 
		\caption{تصویر سمت چپ تصویر اصلی و تصویر سمت راست تصویر نویزی}
		\label{q211}
	\end{figure}

\newpage

	\begin{figure}[!hbt]
		\centering
		\begin{minipage}{.5\linewidth}
			\includegraphics[width = 7cm]{SharifLogo.png}
			\caption{caption 1}
			\label{q3115}
		\end{minipage}%
		\begin{minipage}{.5\linewidth}
			\includegraphics[width = 7cm]{SharifLogo.png}
			\caption{caption 2}
			\label{q3116}
		\end{minipage}
	\end{figure}


		\begin{lstlisting}
			rgb = imread('circles.jpg');
			% imshow(rgb);
			
			gray_image = rgb2gray(rgb);
			% imshow(gray_image);
			
			[centers,radii] = imfindcircles(rgb,[20 25],'ObjectPolarity','dark', ...
			'Sensitivity',0.86,'Method','twostage');
			[centersBright,radiiBright,metricBright] = imfindcircles(rgb,[20 25], ...
			'ObjectPolarity','bright','Sensitivity',0.92,'EdgeThreshold',0.1);
			
			imshow(rgb)
			h = viscircles(centers,radii,'Color','k');
			hBright = viscircles(centersBright, radiiBright,'Color','k');
		\end{lstlisting}
	
	\newpage
	
	\bibliographystyle{plain}
	\bibliography{bibliographicResource}

\end{document} 